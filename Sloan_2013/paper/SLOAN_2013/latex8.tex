%
%  $Description: Author guidelines and sample document in LaTeX 2.09$
%
%  $Author: ienne $
%  $Date: 1995/09/15 15:20:59 $
%  $Revision: 1.4 $
%

\documentclass[times, 10pt,twocolumn]{article}
\usepackage{latex8}
\usepackage{times}
\usepackage{balance}

%\documentstyle[times,art10,twocolumn,latex8]{article}

%-------------------------------------------------------------------------
% take the % away on next line to produce the final camera-ready version
\pagestyle{empty}

%-------------------------------------------------------------------------
\begin{document}

\title{Author Guidelines for ICPR 2008 Proceedings Manuscripts}

\author{Author(s) Name(s)\\
\emph{Author Affiliation(s)}\\
\emph{E-mail}\\
% For a paper whose authors are all at the same institution,
% omit the following lines up until the closing ``}''.
% Additional authors and addresses can be added with ``\and'',
% just like the second author.
%\and
%Second Author\\
%Institution2\\
%First line of institution2 address\\ Second line of institution2 address\\
%SecondAuthor@institution2.com\\
}

\maketitle
\thispagestyle{empty}

\begin{abstract}
The abstract is to be in fully-justified italicized text, at the top of the left-hand column as it is here, below the author information. Use the word "Abstract" as the title, in 12-point Times, boldface type, centered relative to the column, initially capitalized. The abstract is to be in 10-point, single-spaced type, and up to 150 words in length. Leave two blank lines after the abstract, then begin the main text.
\end{abstract}



%-------------------------------------------------------------------------
\Section{Introduction}

All manuscripts must be in English. These
guidelines include complete descriptions of the fonts, spacing, and
related information for producing your proceedings manuscripts.

%-------------------------------------------------------------------------
\Section{Formatting your paper}

All papers must be formatted in Letter size (8.5 x 11 inches). All printed material, including text, illustrations, and charts, must be kept within a print area of 6-1/2 inches (16.51 cm) wide by 8-7/8 inches (22.51 cm) high. Do not write or print anything outside the print area. All \emph{text} must be in a two-column format. Columns are to be 3-1/16 inches (7.85 cm) wide, with a 3/8 inch (0.81 cm) space between them. Text must be fully justified. If you use A4 size paper increase bottom margin so that the printed area matches the description given above.

%-------------------------------------------------------------------------
\Section{Main title}

The main title (on the first page) should begin 1-3/8 inches (3.49 cm) from the top edge of the page, centered, and in Times 14-point, boldface type. Capitalize the first letter of nouns, pronouns, verbs, adjectives, and adverbs; do not capitalize articles, coordinate conjunctions, or prepositions (unless the title begins with such a word). Leave two 12-point blank lines after the title.

%-------------------------------------------------------------------------
\Section{Author name(s) and affiliation(s)}

Author names and affiliations are to be centered beneath the title and printed in Times 12-point, non-boldface type. Multiple authors may be shown in a two- or three-column format, with their affiliations italicized and centered below their respective names. Include e-mail addresses if possible. Author information should be followed by two 12-point blank lines.

%-------------------------------------------------------------------------
\Section{Second and following pages}

The second and following pages should begin 1.0 inch (2.54 cm) from the top edge. On all pages, the bottom margin should be 1-1/8 inches (2.86 cm) from the bottom edge of the page.

%------------------------------------------------------------------------
\Section{Type-style and fonts}

Wherever Times is specified, Times Roman or Times New Roman may be used. If neither is available on your word processor, please use the font closest in appearance to Times. Avoid using bit-mapped fonts if possible. True-Type 1 fonts are preferred.

%-------------------------------------------------------------------------
\Section{Main Text}

Type your main text in 10-point Times, single-spaced. Do not use double-spacing. All paragraphs should be indented 1/4 inch (approximately 0.5 cm). Be sure your text is fully justified-that is, flush left and flush right. Please do not place any additional blank lines between paragraphs.

{\bfseries Figure and table captions} should be 10-point boldface Helvetica (or a similar sans-serif font). Callouts should be 9-point non-boldface Helvetica. Initially capitalize only the first word of each figure caption and table title. Figures and tables must be numbered separately. For example: "Figure 1. Database contexts", "Table 1. Input data". Figure captions are to be centered \emph{below} the figures. Table titles are to be centered \emph{above} the tables.

%Figure and table captions should be 10-point
%Helvetica boldface type as in
%\begin{figure}[h]
%   \caption{Example of caption.}
%\end{figure}

%\noindent Long captions should be set as in
%\begin{figure}[h]
%   \caption{Example of long caption requiring more than one line. It is
%     not typed centered but aligned on both sides and indented with an
%     additional margin on both sides of 1~pica.}
%\end{figure}

%\noindent Callouts should be 9-point Helvetica, non-boldface type.

\Section{First-order headings}

For example, "1. Introduction", should be Times 12-point boldface, initially capitalized, flush left, with one blank line before, and one blank line after. Use a period (".") after the heading number, not a colon.

\SubSection{Second-order headings}

As in this heading, they should be Times 11-point boldface, initially capitalized, flush left, with one blank line before, and one after.

\paragraph{8.1.1. Third-order headings.} Third-order headings, as in this paragraph, are discouraged. However, if you must use them, use 10-point Times, boldface, initially capitalized, flush left, preceded by one blank line, followed by a period and your text on the same line.

%\noindent {\bf 8.1.1. Third-order headings.} Third-order headings, as in this paragraph, are discouraged. However, if you must use them, use 10-point Times, boldface, initially capitalized, flush left, preceded by one blank line, followed by a period and your text on the same line.

%-------------------------------------------------------------------------
\Section{Footnotes}

Use footnotes sparingly (or not at all) and place them at the bottom of the column on the page on which they are referenced. Use Times 8-point type, single-spaced. To help your readers, avoid using footnotes altogether and include necessary peripheral observations in the text (within parentheses, if you prefer, as in this sentence).


%\footnote
%   {%
%     Or, better still, try to avoid footnotes altogether.)
%   }

%------------------------------------------------------------------------
\Section{Copyright forms}

We do not require copyright forms at the time of initial paper submission for review. You must include a signed IEEE copyright release form when you submit your final accepted paper.

%-------------------------------------------------------------------------
\Section{References}

List and number all bibliographical references in 9-point Times,
single-spaced, at the end of your paper. When referenced in the text,
enclose the citation number in square brackets, for example~\cite{ex1}.
Where appropriate, include the name(s) of editors of referenced books.
\balance

%-------------------------------------------------------------------------
\nocite{ex1,ex2}
\bibliographystyle{latex8}
\bibliography{latex8}

\end{document}

